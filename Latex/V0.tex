%%%%%%%%%%%%%%%%%%%%%%%%%%%%%%%%%%%%%%%%%%%%%%%%%%%%%%%%%%%%%%%%%%%%%%%
% Universidade Federal de Santa Catarina             
% Biblioteca Universit�ria                     
%----------------------------------------------------------------------
% Exemplo de utiliza��o da documentclass ufscThesis
%----------------------------------------------------------------------                                                           
% (c)2010 Roberto Simoni (roberto.emc@gmail.com)
%         Carlos R Rocha (cticarlo@gmail.com)
%%%%%%%%%%%%%%%%%%%%%%%%%%%%%%%%%%%%%%%%%%%%%%%%%%%%%%%%%%%%%%%%%%%%%%%
\documentclass{ufscThesis} % Definicao do documentclass ufscThesis


%----------------------------------------------------------------------
% Pacotes usados especificamente neste documento
\usepackage{graphicx} % Possibilita o uso de figuras e gr�ficos
\usepackage{color}    % Possibilita o uso de cores no documento


\usepackage[brazil]{babel}

\usepackage{nomencl} % Pacote necess�rio para a lista de siglas.

\usepackage{epstopdf}
\usepackage{mathcomp}
\usepackage{gensymb}
\usepackage[enable-survey]{pdfpages}    %para incluir pdf   [enable-survey]

\usepackage{booktabs} % Para Tabelas
\usepackage{subfig}
\usepackage{float}
\usepackage{url} % Para incluir links

%----------------------------------------------------------------------
% Comandos criados pelo usu�rio
\newcommand{\afazer}[1]{{\color{red}{#1}}} % Para destacar uma parte a ser trabalhada

%----------------------------------------------------------------------
% Identificadores do trabalho
% Usados para preencher os elementos pr�-textuais
\instituicao[a]{Universidade Federal de Santa Catarina} % Opcional
\departamento[a]{Biblioteca Universit�ria}
\curso[o]{Engenharia El�trica}
\documento[o]{Exemplo} % Opcional (Tese � o padr�o)
\titulo{UMA PROPOSTA AVALIA��O INTEGRADA DO USO ENERG�TICO DE RES�DUOS
S�LIDOS URBANOS POR MEIO DA TEORIA DE UTILIDADE MULTIATRIBUTO
(MAUT)}
%\subtitulo{Vers�o B�sica} % Opcional
\autor{Arthur Mendon�a Quinhones Siqueira}
\grau{Gradua��o em Engenharia El�trica}
\local{Florian�polis} % Opcional (Florian�polis � o padr�o)
\data{Dia}{Mar�o}{2017}
\orientador[Orientadora]{Coelho}
\coorientador[Coorientadora]{Sica}
\coordenador[Coordenador]{Pacheco}



\numerodemembrosnabanca{6} % Isso decide se haver� uma folha adicional
\orientadornabanca{nao} % Se faz parte da banca definir como sim
\coorientadornabanca{sim} % Se faz parte da banca definir como sim
\bancaMembroA{Presidente da banca} %Nome do presidente da banca
\bancaMembroB{Segundo membro}      % Nome do membro da Banca
\bancaMembroC{Terceiro membro}     % Nome do membro da Banca
\bancaMembroD{Quarto membro}       % Nome do membro da Banca
%\bancaMembroE{Prof. quinto membro}       % Nome do membro da Banca
%\bancaMembroF{Prof. sexto membro}        % Nome do membro da Banca
%\bancaMembroG{Prof. s�timo membro}       % Nome do membro da Banca



%\dedicatoria{Escrever bonitinho}{Arthur}






\epigrafe{Aprender � a �nica coisa de que a mente nunca se cansa, nunca tem medo e nunca se arrepende.}
{Leonardo da Vinci}


\textoResumo {Escrever aqui o Resumo}

\palavrasChave {Palavras Chave}

 
\textAbstract {Escrever aqui o Abstract}
\keywords {KEYWORDS}

%----------------------------------------------------------------------
% In�cio do documento                                
\begin{document}
%--------------------------------------------------------
% Elementos pr�-textuais
\capa  
\folhaderosto[comficha] % Se nao quiser imprimir a ficha, � s� n�o usar o par�metro
\folhaaprovacao
\paginadedicatoria
\paginaagradecimento
\paginaepigrafe
\paginaresumo
\paginaabstract
%\pretextuais % Substitui todos os elementos pre-textuais acima
\listadefiguras % as listas dependem da necessidade do usu�rio
\listadetabelas 
\listadeabreviaturas
\listadesimbolos
\sumario
%--------------------------------------------------------

% Elementos textuais



\chapter{Introdu��o}

\section{Problem�tica}
O consumo crescente, desenfreado e em excesso nos centros urbanos tem como consequ�ncia uma  grande quantidade de res�duos s�lidos urbanos. Este padr�o de consumo tem como consequ�ncia uma sobrecarrega dos sistemas de coleta e disposi��o do "lixo" em local adequado, de forma que o Estado � obrigado � direcionar grande volume de recursos para este fim. 
			
Este consumismo exige uma boa parcela da gera��o de energia para produ��o dos produtos. Observa-se uma tendencia na dire��o de se evitar a constru��o de grandes empreendimentos hidroel�tricos e t�rmicos, devido � seus impactos socio ambientais. Assim, a oferta de energia gerada com fontes intermitentes, e�lica e solar, tem se mostrado como uma alternativa de incremento do sistema de gera��o de energia el�trica. 

Entretanto, para o sistema de distribui��o de energia operar com confiabilidade, exige-se uma complementaridade com fontes de gera��o cont�nuas, de forma a garantir o atendimento � carga nos per�odos de pouco vento e radia��o solar. 

O crescimento da inser��o de fontes intermitentes tem preocupado as companhias de distribui��o pois n�o h� uma garantia absoluta do atendimento � carga e, al�m disso, elas [podem gerar?] geram instabilidade nos sistemas de distribui��o.

Como a demanda de energia el�trica tem crescido mais do que a oferta, o pre�o deste bem para o consumidor residencial fica cada vez mais alto. Uma das consequ�ncias do alto pre�o do kilo-Watt-hora � o incentivo da inser��o de fontes alternativas de gera��o de energia el�trica, conectadas � rede de distribui��o. Tal fato deve-se ao sistema de compensa��o implementado pela resolu��o 482 da Ag�ncia Nacional de Energia El�trica (ANEEL) no ano de 2012.

Neste contexto, deseja-se avaliar as possibilidades do uso de biog�s proveniente dos res�duos s�lidos urbanos para gera��o distribuida de energia el�trica, bem como incentivar a coleta seletiva dos materiais reaproveit�veis e recicl�veis. 

O do trabalho tem como objetivo incentivar o desenvolvimento local com gest�o social atrav�s da educa��o ambiental, buscando descentralizar a gera��o de energia el�trica e otimizar o manejo dos res�duos s�lidos urbanos.

\section{Objetivo Geral}
Proposta de avalia��o do uso energ�tico do biog�s proveniente dos RSU para GD por meio de metodologia multicriterial de apoio � decis�o de modo � proporcionar a adequa��o de munic�pios � PNRS atrav�s do desenvolvimento local com da gest�o social.

\section{Objetivos Espec�ficos}
\begin{itemize}
\item Estudar/Buscar/Propor alternativas para descentraliza��o da coleta, manejo, tratamento e disposi��o dos RSU
\item Avaliar o impacto da micro/mini gera��o t�rmica conectada ao sistema de distribui��o de energia el�trica
\item Explorar/propor/apresentar/estudar/incorporar/implementar o uso de metodologias participativas para o planejamento integrado de recursos
\item Desenvolver um estudo de caso por meio de um software de simula��o computacional
\end{itemize}

\section{Justificativa}
Pendente
\section{Hip�teses}
\begin{itemize}
\item Aumento da confiabilidade do sistema de distribui��o atrav�s gera��o t�rmica distribu�da.
\item Redu��o de custos com coleta, transporte, tratamento e disposi��o dos Res�duos S�lidos Urbanos
\item Proporcionar efici�ncia energ�tica atrav�s de processos integrados
\item Desenvolvimento local atrav�s da gest�o social e educa��o ambiental
\end{itemize}

\section{Metodologia}
Metodologia de desenvolvimento do trabalho ou metodologia usada no trabalho?

\section{Etapas e Organiza��o do trabalho}

\begin{enumerate}
\item Revis�o Bibliogr�fica:
			Conceitos b�sicos
			Pol�ticas p�blicas voltadas aos res�duos s�lidos e gera��o distribuida. 
			Gest�o e Gerenciamento de RS
			Estado da arte - Tecnologias de obten��o do Biog�s
			Estado da arte - Tecnologias de gera��o distribuida com Biog�s
			Metodologias Multicrit�rio de Apoio a Decis�o
			Desenvolvimento local, gest�o social
			Pesquisa-A��o		
\item Modelagem da metodologia:
			EPA
			determina��o dos pontos de vista fundamentais
			determina��o dos pontos de vista elementares
			determina��o dos descritores
\item Implementa��o do modelo
\item Estudo de caso
			Usar a ferramenta desenvolvida para avaliar o uso do biog�s proveniente dos res�duos s�lidos urbnanos para gera��o de energia t�rmica em diferentes contextos. 
\end{enumerate}			

\section{Resultado Esperado}
\begin{itemize}
\item Modelar, implementar uma ferramenta de apoio � decis�o baseado em uma metodologia multicriterial para estudos de caso do uso de biog�s proveniente dos res�duos s�lidos urbanos na gera��o distribuida de energia el�trica. 
\item Usar no modelo crit�rios de natureza  ambiental, economica, t�cnica, social e pol�tica. 
\item Apresentar resultados dos estudos de caso de modo � compor uma alternativa favor�vel a descentraliza��o do manejo de res�duos s�lidos urbanos e gera��o t�rmica distribuida.
\end{itemize}
\chapter{Diretrizes Pol�ticas e Instrumentos}

Este capitulo vai mostrar quais s�o as diretrizes pol�ticas que fundamentam o trabalho

\section{Pol�tica Nacional dos Res�duos S�lidos}

A Pol�tica Nacional dos Res�duos S�lidos (PNRS) foi institu�da pela Lei n� 12.305, de 2 de agosto de 2010. Sua fun��o � reunir um conjunto de princ�pios, objetivos, instrumentos, metas e a��es adotados pelo governo federal, isoladamente ou em regime de coopera��o com estados, distrito federal, munic�pios, ou particulares, com vistas � gest�o integrada e ao gerenciamento ambientalmente adequado dos RS 
\citep{pnrs}.

Seus princ�pios e objetivos, contidos nos artigos 6 e 7 do segundo cap�tulo, s�o os fatores fundamentais da motiva��o e elabora��o deste trabalho. Dentre eles destacam-se os seguintes princ�pios: 

\begin{itemize}

\item A vis�o sistemica, na gest�o dos res�duos s�lidos, que considere as vari�veis ambiental, social, cultural, econ�mica, tecnol�gica e de sa�de p�blica
\item Coopera��o entre as diferentes esferas da sociedade setor p�blico, o setor empresarial e demais segmentos da sociedade
\item O reconhecimento do res�duo s�lido reutiliz�vel e recicl�vel como um bem econ�mico e de valor social, gerador de trabalho e renda e promotor de cidadania
\item Respeito �s diversidades locais e regionais
\item Direito da sociedade � informa��o e ao controle social
\item A razoabilidade e a proporcionalidade

\end{itemize}

Bem como os seguintes objetivos:

\begin{itemize}

\item Est�mulo � ado��o de padr�es sustent�veis de produ��o e consumo de bens e servi�os
\item Gest�o integrada de res�duos s�lidos
\item Capacita��o t�cnica continuada na �rea de RS
\item Regularidade, continuidade, funcionalidade e universaliza��o da presta��o dos servi�os p�blicos de limpeza urbana e manejo de res�duos s�lidos, com ado��o de mecanismos gerenciais e econ�micos que assegurem a recupera��o dos custos dos servi�os prestados, como forma de garantir sua sustentavilidade operacional e financeira, observada a lei n�11.445, de 2007.
\item Incentivo ao desenvolvimento de sistemas de gest�o ambiental e empresarial voltados para melhoria dos processos produtivos e ao reaproveitamento dos residuos s�lidos, incluindo a recupera��o e aproveitamento energ�tico.

\end{itemize}
\include{Capitulos/06/chapter_06}

\chapter{Metodologia Multicriterial de apoio � Decis�o}

%Go to \url{http://www.uni.edu/~myname/best-website-ever.html} for my website.
%
%
%
%\section{Defini��o das Vari�veis do Documento}
%\section{Gera��o dos Elementos Pr�-Textuais}
%\subsection{Elementos introdut�rios}
%\subsubsection{Capa e Folha de Rosto}
%\subsubsection{Folha de Aprova��o}
%\subsubsection{Dedicat�ria, Agradecimento e Ep�grafe}
%\subsubsection{Resumo e Abstract}
%\subsection{Listas}
%\subsubsection{Sum�rio}
%\subsubsection{Listas de Figuras e Tabelas}
%\subsubsection{Lista de Abreviaturas e Siglas}
%\subsubsection{Lista de S�mbolos}
%\subsubsection{Outras Listas} 

\chapter{Simula��o e Estudo de Caso}

Texto do c�pitulo


%\section{Defini��o das Vari�veis do Documento}
%\section{Gera��o dos Elementos Pr�-Textuais}
%\subsection{Elementos introdut�rios}
%\subsubsection{Capa e Folha de Rosto}
%\subsubsection{Folha de Aprova��o}
%\subsubsection{Dedicat�ria, Agradecimento e Ep�grafe}
%\subsubsection{Resumo e Abstract}
%\subsection{Listas}
%\subsubsection{Sum�rio}
%\subsubsection{Listas de Figuras e Tabelas}
%\subsubsection{Lista de Abreviaturas e Siglas}
%\subsubsection{Lista de S�mbolos}
%\subsubsection{Outras Listas}

\include{Capitulos/05/chapter_05}



\bibliographystyle{ufscThesis/ufsc-alf}
\bibliography{bibliografia}


%--------------------------------------------------------
% Elementos p�s-textuais

%\apendice

%\include{Apendices/A/apendice_A}



%\anexo


%\include{Anexos/A/anexo_A}


\end{document}